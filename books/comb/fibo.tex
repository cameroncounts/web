\documentclass[12pt]{article}

\usepackage{amsfonts}

\newtheorem{theorem}{Theorem}[section]
\newtheorem{proposition}[theorem]{Proposition}
\newtheorem{lemma}[theorem]{Lemma}
\newtheorem{definition}[theorem]{Definition}
\newcommand{\head}[1]{\medbreak\noindent\textit{#1}\ }

\begin{document}

\title{Fibonacci notes}
\author{Peter J. Cameron and Dima G. Fon-Der-Flaass}
\date{}
\maketitle

\begin{abstract}
These notes put on record part of the contents of a conversation
the first author had with John Conway in November 1996, concerning some
remarkable properties of the Fibonacci numbers discovered by Clark
Kimberling~\cite{kimberling} and by Conway himself. Some of these
properties are special cases of much more general results, while
others are specific to the Fibonacci sequence; some are proved,
while others are merely observation (as far as we know). The first
four sections are purely expository.
The last two sections on numeration systems are the work of the
second author. We am grateful to members of the Combinatorics Study Group
at QMW, especially Julian Gilbey, for discussions and help with the details.
\end{abstract}

\section{Fibonacci numbers}

The \emph{Fibonacci number} $F_n$, for positive integer $n$, can be
defined as the number of ways of writing $n$ as the sum of a
sequence of terms, each equal to $1$ or $2$. So, for example, $4$
can be expressed in any of the forms
\[2+2=2+1+1=1+2+1=1+1+2=1+1+1+1,\]
so $F_4=5$.

The most important property of the Fibonacci numbers is that they
satisfy the recurrence relation
\[F_n=F_{n-1}+F_{n-2}\]
for $n\ge3$. For consider all the sequences of $1$s and $2$s
with sum~$n$. Divide the sequences into two classes according to
whether the last term is $1$ or $2$. There are $F_{n-1}$ sequences
in the first class, since the terms except the last sum to~$n-1$.
Similarly, there are $F_{n-2}$ sequences in the second class.
Since the classes don't overlap, and every sequence lies in one
of them, the recurrence relation follows.

This recurrence, together with the initial values $F_1=1$ and
$F_2=2$, determines $F_n$ for all~$n$.

We can extend the definition of $F_n$ to $n=0$ in a natural way:
the empty sequence is the only one with sum~$0$, so $F_0=1$.
It is possible to define $F_n$ for negative $n$ so that the
recurrence is satisfied, but there is no natural counting
interpretation of these numbers.

Standard arguments give the formula
\[F_n=\frac{1}{\sqrt{5}}\left(\left(\frac{1+\sqrt{5}}{2}\right)^{n+1}
-\left(\frac{1-\sqrt{5}}{2}\right)^{n+1}\right).\]

Since $\alpha=(1+\sqrt{5})/2=1.618\ldots>1$ and
$\beta=(1-\sqrt{5})/2=-0.618\ldots>-1$, we see that $F_n$ is the nearest
integer to $(1/\sqrt{5})((1+\sqrt{5})/2)^{n+1}$. In particular,
$F_{n+1}/F_n$ tends to the limit $(1+\sqrt{5})/2$ as $n\to\infty$.

Note that $\alpha-1=-\beta$ is the \emph{golden ratio} $\tau$, the point
of division of a unit interval with the property that the ratio of the
larger part to the whole is equal to the ratio of the smaller part to the
larger.

\section{The Fibonacci successor function}
\label{s:fsf}

In this section we define a function on the positive integers
which has the property that it maps each Fibonacci number to
the next. It depends on the following property of Fibonacci
numbers:

\begin{theorem}
Every positive integer $n$ has a unique expression in the form
\[n=F_{i_1}+F_{i_2}+\cdots+F_{i_k},\]
where $i_{j+1}\ge i_j+2$ for $j=1, \ldots, k-1$; in other words,
as the sum of a set of Fibonacci numbers with no two consecutive.
\end{theorem}

We call the expression in the theorem the \emph{Fibonacci
representation} of~$n$.

To prove this, we use a simple fact about Fibonacci numbers,
easily demonstrated by induction:

\begin{lemma}
The sum of alternate Fibonacci numbers ending with $F_k$ is
$F_{k+1}-1$.
\end{lemma}

\head{Proof} The inductive step goes from $k$ to $k+2$: if we
assume the result for $k$, and add $F_{k+2}$ to both sides of
the equation, then on the right we obtain $F_{k+3}-1$. So it is
necessary to start the induction separately for odd and even $k$,
by observing that $F_1=F_2-1$ and $F_2=F_3-1$.

\medbreak

Now given an integer $n$, let $F_k$ be the largest Fibonacci
number not exceeding $n$. Then any representation of $n$ as the
sum of Fibonacci numbers with no two consecutive must include
$F_k$, since by the Lemma if we omit $F_k$ the greatest we can
get is $F_k-1$. Now, by induction, $n-F_k$ has a unique Fibonacci
representation; and this representation cannot include $F_{k-1}$,
since $n<F_{k+1}=F_k+F_{k-1}$. So we have constructed the unique
Fibonacci representation of $n$.

Now we define the \emph{Fibonacci successor function} $\sigma$
as follows: if
\[n=F_{i_1}+F_{i_2}+\cdots+F_{i_k}\]
is the Fibonacci representation of~$n$, then
\[\sigma(n)=F_{i_1+1}+F_{i_2+1}+\cdots+F_{i_k+1}.\]
Note that $\sigma(F_k)=F_{k+1}$, and that $m$ is the Fibonacci
successor of some (unique) positive integer if and only if $F_1$
does not occur in the Fibonacci representation of $m$.

Note also that $n+\sigma(n)=\sigma^2(n)$ for any $n$, where
$\sigma^2(n)$ denotes $\sigma(\sigma(n))$. So, in order to
apply the successor function repeatedly, we only have to apply
it once and then use the Fibonacci recurrence relation.

We call the positive integer $n$ a \emph{Fibonaacci successor} if
$n=\sigma(m)$ for some $m$. This holds if and only if the
Fibonacci representation of $n$ does not contain $F_1$.

We oserved in the last section that each Fibonacci number is
approximately $\alpha$ times the preceding one, where $\alpha=
(1+\sqrt{5})/2$. The convergence is exponentially rapid. A small
amount of analysis of geometric progressions in fact shows the
following. Given a positive integer $n$, let $\rho(n)$ be the
integer nearest to $n\alpha$.

\begin{theorem}
For any positive integer~$n$, either $\sigma(n)=\rho(n)$, or
$\sigma(n)=\rho(n)+1$. The first alternative holds if $n$ is a
Fibonacci successor.
\label{t:approx}
\end{theorem}

We can extend the domain of the Fibonacci successor function to
include zero, in a natural way. Since the Fibonacci representation
of zero is empty, we take $\sigma(0)=0$.

\section{The table}
\label{s:table}

We consider the following table. (Ignore for the moment the
two columns at the left.)
\[
\begin{array}{|r|r|rrrrrrrrr|}
\hline
0&1&1&2&3&5&8&13&21&34&\ldots\\
\hline
1&3&4&7&11&18&29&47&76&123&\ldots\\
\hline
2&4&6&10&16&26&42&68&110&178&\ldots\\
\hline
3&6&9&15&24&39&63&102&165&267&\ldots\\
\hline
4&8&12&20&32&52&84&136&220&356&\ldots\\
\hline
5&9&14&23&37&60&97&157&254&411&\ldots\\
\hline
6&11&17&28&45&73&118&191&309&500&\ldots\\
\hline
7&12&19&31&50&81&131&212&343&555&\ldots\\
\hline
8&14&22&36&58&94&152&246&398&644&\ldots\\
\hline
9&16&25&41&66&107&173&280&453&733&\ldots\\
\hline
10&17&27&44&71&115&186&301&487&788&\ldots\\
\hline
\end{array}
\]

The table is constructed as follows. The first row contains the
Fibonacci numbers. As we already saw, these are produced by
starting with~$1$ and applying the Fibonacci successor function
repeatedly.

The first element in the next row is the smallest number which
has not been encountered previously (which happens to be~$4$).
Then we apply the successor function repeatedly to it to generate
the row. As we also saw, we only need to apply the successor
function once:
\[4=1+3=F_1+F_3,\]
so
\[\sigma(4)=F_2+F_4=2+5=7.\]
Then the remaining elements can be found from the Fibonacci
recurrence relation: next is $7+4=11$, then $11+7=18$, and so on.

Now we repeat this for each succeeding row: the first entry is the
smallest number not used in the table so far, and the rest of its
row is obtained by applying the successor function repeatedly
(or by applying the successor function once and then the recurrence
relation).

Clearly the first element in any row is not itself the
Fibonacci successor of anything. So the first elements in
the rows are the numbers whose Fibonacci representations include
$F_1$, arranged in order. We see that every positive integer
occurs exactly once in the table: the integer
\[n=F_{i_1}+F_{i_2}+\cdots+F_{i_k}\]
occurs in column $i_1$ of the row with first element
\[m=F_1+F_{i_2-i_1+1}+\cdots+F_{i_k-i_1+1}.\]

Let $T_{i,j}$ denote the entry in row~$i$ and column~$j$ of the table.
(For technical reasons, we take the first row to have number~$0$).
Various remarkable properties hold: we now give some of these.

\begin{theorem}\label{th3_1}
Let $a_1$ and $a_2$ be positive integers, and define a sequence
$(a_n)$ by the Fibonacci recurrence: that is, $a_{n+2}=a_n+a_{n+1}$
for $n\ge1$. Then there exist $k,l,m$ such that $a_{m+n}=T_{k,l+n}$
for all $n\ge0$. In other words, every sequence generated by
the Fibonacci recurrence occurs, from some point on, in the table.
\end{theorem}

\head{Proof} It is enough to show that $a_{n+1}=\sigma(a_n)$ for
some $n$: for the number $a_n$ occurs in the table (by our
previous remark) and the given sequence agrees with that row from
that point on.

Now the solution to any Fibonacci recurrence is given by
\[a_n=A\alpha^n+B\beta^n\]
for some $A$ and $B$, where $\alpha$ and $\beta$ are as in
Section~1. Since $|\beta|<1$, for sufficiently large $n$ we see
that $a_{n+1}=\rho(a_n)$, the nearest integer to $a_n\alpha$.
So, by Theorem~\ref{t:approx}, it is enough to find a sufficiently
large $n$ such that $a_n$ is a Fibonacci successor.

Suppose that $a_n$ is not a Fibonacci successor, say
\[a_n=F_1+F_k+\cdots,\]
with $k\ge3$. Then by Theorem~\ref{t:approx}, we have
\[a_{n+1}=\rho(a_n)=\sigma(a_n)-1=F_1+F_{k+1}+\cdots.\]
But then
\[a_{n+2}=a_n+a_{n+1}=F_2+F_{k+2}+\cdots\]
is a Fibonacci suuccessor, as required.

\head{Remark} If two sequences satisfying the Fibonacci recurrence
agree from some point on, then (by extrapolting back) they agree
throughout their entire ength. So, in Theorem~3.1, we may assume
that either $l$ or $m$ is equal to~$1$.

\medbreak

Let $r_n$ be the number of the row containing the positive integer~$n$.
The sequence $(r_n)$ begins
\[0,0,0,1,0,2,1,0,3,2,1,4,0,5,3,2,6,1,7,4,0,\ldots\]
The lengths of the gaps between successive zeros are Fibonacci numbers
(starting with $F_0=1$). The numbers in the gap of length $F_k$ are
a permutation $\pi_k$ of $\{1,2,\ldots, F_k-1\}$, and the permutation 
$\pi_{k+1}$ is obtained from $\pi_k$ by inserting the additional numbers
in appropriate gaps. 

\medbreak

We can imagine that the table is extrapolated backwards using
the recurrence: that is, $T_{i,0}=T_{i,2}-T_{i,1}$ and
$T_{i,-1}=T_{i,1}-T_{i,0}$ for all $i\ge0$.

\begin{theorem}
We have $T_{i,{-1}}=i$ and $T_{i,0}=\sigma(i)+1$. Moreover,
$T_{i,1}=\sigma(T_{i,0})-1$.
\end{theorem}

\head{Proof} The function $\sigma$ is strictly monotonic, and the
entries in column~$1$ of the table are (by construction) increasing.
So the numbers
\[\sigma^{-1}(\sigma^{-1}(T_{i,1}+1)-1)\]
are strictly increasing. So it suffices to prove that every
non-negative integer $n$ can be expressed in the form
\[n=\sigma^{-1}(\sigma^{-1}(m+1)-1)\]
for some $m$ which is not a Fibonacci successor, and that,
conversely, if $m$ is not a Fibonacci successor, then it can
be written in the form
\[m=\sigma(\sigma(n)+1)-1\]
for some $n$.

For the first, take any $n$ and write its Fibonacci representation.
There are two cases. If $F_1$ occurs (that is, $n$ is not a
Fibonacci successor), then, say,
\[n=F_1+\cdots+F_{2l-1}+F_k+\cdots,\]
with $k>2l+1$. Then
\[\sigma(n)+1=F_{2l+1}+F_{k+1}+\cdots,\]
whence
\[\sigma(\sigma(n)+1)-1=F_1+\cdots+F_{2l+1}+F_{k+2}+\cdots,\]
which is not a Fibonacci successor. On the other hand, if $F_1$
does not occur in the representation of $n$, then say
\[n=F_2+\cdots+F_{2l}+F_k+\cdots,\]
with $k\ge2l+2$ (and possibly $l=0$). Then
\[\sigma(n)+1=F_1+F_3+\cdots+F_{2l+1}+F_{k+1}=\cdots,\]
and
\[\sigma(\sigma(n)+1)-1)=F_1+F_4+\cdots+F_{2l+2}+F_{k+2}+\cdots,\]
which again is not a Fibonacci successor.

For the converse, suppose that
\[m=F_1+F_3+\cdots+F_{2l+1}+F_k=\cdots,\]
where $k>2l+3$. Then
\[m+1=F_{2l+2}+F_k+\cdots,\]
so that
\[\sigma^{-1}(m+1)=F_{2l+1}+F_{k-1}+\cdots.\]
Then
\[\sigma^{-1}(m+1)-1=F_2+\cdots+F_{2l}+F_{k-1}+\cdots,\]
and so
\[\sigma^{-1}(\sigma^{-1}(m+1)-1)=F_1+\cdots+F_{2l-1}+F_k=\cdots.\]
(It is possible that the initial terms $F_1+\cdots+F_{2l-1}$ are
absent, if $l=0$.)

\medbreak

So the entries in column~$-1$ conveniently label the rows by
non-negative integers.

\section{Phyllotaxis}

Fibonacci numbers, and the golden ratio, are popularly associated
with the growth of plants. In order that the leaves of a growing
plant should not shade the leaves below them, each new leaf
should grow so as to make an angle of $\tau$ of a circle with
the previous one, where $\tau=(\sqrt{5}-1)/2$ is the golden ratio.
See the discussion in Coxeter~\cite{coxeter}, Chapter~11.

Let us consider the leaves on such an idealised plant. Suppose
that the stem has unit circumference, and that leaf number zero
grows at the reference point $0$. Then the position of leaf number
$n$ is at $\{n\tau\}=n\tau-\lfloor n\tau\rfloor$, the fractional
part of $n\tau$. When it emerges, the circle is already divided
into $n$ intervals by the existing leaves. Suppose that there
are $a_n$ intervals between the zeroth and the $n$th leaf when it
emerges, and $b_n$ intervals between the $n$th leaf and the zeroth.
(Thus $a_n+b_n=n+1$.) What can be said about the ordered pairs
$(a_n,b_n)$?

It is known that the ratio of consecutive Fibonacci numbers is a
close approximation to the golden ratio. More precisely,
\[\lim_{k\to\infty}F_k/F_{k+1}=\tau,\]
and the ratio $F_k/F_{k+1}$ is a better approximation to $\tau$
than any rational with smaller denominator; moreover, the ratio
is alternately greater and less than its limit. Thus, when leaf
number $n=F_{k+1}$ emerges, we have either $a_n=1$ (if $k$ is odd)
or $b_n=1$ (if $k$ is even). Thus,
\[\hbox{If }n=F_{k+1}\hbox{ then } (a_n,b_n)=
\cases{(1,n)& if $k$ is odd,\cr (n,1)& if $k$ is even.\cr}\]
Moreover, the Fibonacci numbers are the only numbers with this
property: that is, if $n$ is not a Fibonacci number then
$a_n,b_n>1$.

We find the following values:
\[
\begin{array}{|r|r|r|}
\hline
n & a_n & b_n \\
\hline
1  &  1  &  1  \\
2  &  1  &  2  \\
3  &  3  &  1  \\
4  &  2  &  3  \\
5  &  1  &  5  \\
6  &  5  &  2  \\
7  &  3  &  5  \\
8  &  8  &  1  \\
9  &  5  &  5  \\
10  &  2  &  9  \\
11  &  9  &  3  \\
12  &  5  &  8  \\
13  &  1  &  13  \\
14  &  10  &  5  \\
15  &  5  &  11  \\
16  &  15  &  2  \\
17  &  9  &  9  \\
18  &  3  &  16  \\
19  &  15  &  5  \\
20  &  8  &  13  \\
21  &  21  &  1  \\
22  &  13  &  10  \\
23  &  5  &  19  \\
24  &  20  &  5  \\
25  &  11  &  15  \\
\hline
\end{array}
\]

Now observe what happens as $n$ runs along a row of our master
table. We already noted that, if $n$ runs through the Fibonacci
numbers, then one of $a_n$ and $b_n$ is equal to $1$, and this
number bounces from side to side. Empirically, something similar
happens for any row, except that the `bouncing number' is not~$1$
for any other row. In fact, the values of the bouncing number
$t$ are as shown in the following table:
\[
\begin{array}{|r|r|}
\hline
n & t \\
\hline
0 & 1 \\
1 & 3 \\
2 & 2 \\
3 & 5 \\
4 & 8 \\
5 & 5 \\
6 & 9 \\
7 & 5 \\
8 & 10 \\
9 & 15 \\
10 & 9 \\
\hline
\end{array}
\]
The bouncing number seems to be often but not always a Fibonacci
number. We do not know how to explain these patterns!

\section{Recurrent numeration systems}

This section gives a wide generalisation of the table of sequences
satisfying the Fibonacci relation.

\newcommand{\N}{\mathbb{N}} \newcommand{\Z}{\mathbb{Z}} 
\newcommand{\e}{\varepsilon}
We let $\N$ denote the natural numbers, starting at zero, and
$\N^+=\N\setminus\{0\}$ the positive integers. Also,
$\mathcal{P}$ is the set of all nonempty finite subsets of $N$ ordered
lexicographically. We shall often identify elements of $\mathcal{P}$ with
finite binary words: every set $X\in\mathcal{P}$ is identified with the word
$\e_m\ldots\e_0$ where $m=\max X$, and $\e_i=1$ whenever $i\in X$.
Thus the words corresponding to elements of $\mathcal{P}$ always 
begin with $1$.

By a general \emph{numeration system} (NS) we mean any infinite subset
$\mathcal{S}\subseteq\mathcal{P}$ together with the uniquely determined 
order-preserving bijection $n:\mathcal{S}\rightarrow\N^+$.

We now give some examples.

\head{Example 1.} The most trivial example: $\mathcal{S}$ is the collection of
all one-element sets. Then every
positive integer $n$ is the image of the word $B_n=10\ldots 0$ ($n$ zeros).

\head{Example 2} The most common example: the binary system.
$\mathcal{S}=\mathcal{P}$, all non-empty finite sets. The positive
integer $n$ is the image of its own base~$2$ representation.

\head{Example 3} The example which has motivated this section: the Fibonacci
numbering system. $\mathcal{S}$ is the collection of all finite sets
containing no two consecutive numbers.

\medbreak

A numeration system $(\mathcal{S},n)$ is called {\it based}
if there exists a {\it base sequence} $B(\mathcal{S})=(b_0,b_1,\ldots)$
of positive natural numbers such that
$$n(X)=\sum_{i\in X}b_i$$
for every $X\in\mathcal{S}$.

In \emph{Example 1}, the base sequence is just the sequence of natural
numbers $1,2,\ldots$. In \emph{Example 2}, it is the sequence of powers
of $2$, while in \emph{Example 3}, it is the Fibonacci sequence (this
follows from the Fibonacci representation in Section~\ref{s:fsf}).

\medbreak

A NS $(\mathcal{S},n)$ is called {\it tree-like} if it satisfies the following
three properties (here we look at $\mathcal{S}$ as a set of binary words):
\begin{description}
\item{(T1)} $1\in\mathcal{S}$;
\item{(T2)} if $w\in\mathcal{S}$ then $w0\in\mathcal{S}$;
\item{(T3)} every nonempty initial segment of $w\in\mathcal{S}$ belongs to 
$\mathcal{S}$.
\end{description}

The set $\mathcal{S}$ with the set of arcs $(w,w\e)$, $\e\in\{0,1\}$, forms a
directed tree rooted at the vertex $1$. Label every arc $(w,w\e)$ by
$\e$. Now, if we add another vertex $0$, and an arc $(0,1)$ labelled by $1$,
we get a rooted tree $T(\mathcal{S})$, and every element of $\mathcal{S}$
is the sequence of labels on some path beginning at $0$. Every vertex
$w\neq 0$ has outdegree $1$ or $2$, and the outgoing arcs are labelled by
$0$, or by $0$ and $1$. If we start at $0$ and choose an arc with label $1$
whenever possible, we get

\begin{lemma}\label{l1}
Let $\mathcal{S}$ be a tree-like NS. There exists an infinite binary sequence
$M=M(\mathcal{S})=m_1m_2m_3\ldots$, $m_1=1$, the {\em maximal sequence} of 
$\mathcal{S}$, such that every its initial segment $M_k=m_1\ldots m_k$
is the lexicographically maximal $k$-digit word in $\mathcal{S}$.
\end{lemma}

The arcs labelled by $0$ determine a partition of $\mathcal{S}$ into infinite
sequences $C_0,C_1,\ldots$ numbered in the increasing order of their
initial elements. Let $C_i=(v_{i1},v_{i2},\ldots)$. Then the numbers
$c_{ij}=n(v_{ij})$ form the \emph{table} of the NS $(\mathcal{S},n)$.

\medbreak

Our three examples are tree-like.

In \emph{Example 1}, we have $M=1000\ldots$, and $(b_i)=(1,2,3,4,\ldots)$.
The table is a single row $(1,2,3,4,\ldots)$. In \emph{Example 2},
$M=1111\ldots$; and $b_i=2^i$. Entries of the table are $c_{ij}=(2i+1)2^{j-1}$,
for $i\geq 0$, $j\geq 1$: that is, the rows start with the odd numbers, and
each entry is double the one to its left.

In \emph{Example 3}, the base sequence $(b_i)=(1,2,3,5,8,\ldots)$
is the Fibonacci sequence: $b_0=1$, $b_1=2$, $b_n=b_{n-1}+b_{n-2}$.
Finally, $M=101010\ldots$. The table is the one given in Section~\ref{s:table}.

\medbreak

In the other direction, a numeration system can be constructed as follows.

Take any infinite binary sequence $M=m_1m_2\ldots$ beginning with $m_1=1$;
for $i=0,1,\ldots$ let $M_i$ be its initial segment of length $i$, and
define $M_i'=M_{i-1}0$ for those $i>0$ for which $m_i=1$;

Let $\mathcal{S}$ be the set of words $w$ which begin with $1$ and can be
represented in the form $w=v_1'\ldots v_s'v$, $s\geq 0$, where $v_i'$ are
some words $M_j'$, and $v$ is one of $M_i$ (if it exists, such representation
is unique). We have $M=M(\mathcal{S})$ (cf.\ Lemma \ref{l1}).

The base sequence of $\mathcal{S}$ is defined recursively by $b_0=1$ and
\begin{equation}
b_{n}=m_1b_{n-1}+m_2b_{n-2}+\ldots+m_nb_0+1.\label{eq0}
\end{equation}

And now, at last, the first non-trivial result of this section.

\begin{theorem}\label{t1}
Every numeration system which is both tree-like and based is
constructed by the above recipe from an infinite binary sequence $M$.

Conversely, every set $\mathcal{S}$ so constructed is a tree-like based
numeration system.
\end{theorem}

\head{Proof} The theorem immediately follows from three claims:

(1) If $\mathcal{S}$ is a based tree-like NS (for short, BTNS) with
$M=M(\mathcal{S})$ as in Lemma \ref{l1} then its base sequence is given
by the formula (\ref{eq0}).

(2) There exists at most one BTNS with any given base sequence.

(3) The system $\mathcal{S}$ defined in the theorem is a BTNS with the base
sequence given by (\ref{eq0}).

\head{Proof of (1)} The minimal word in $\mathcal{S}$ is 1: this implies that
$b_0=1$. For every $k\geq 1$, the maximal $k$-letter word in $\mathcal{S}$ is
$M_k$, by Lemma \ref{l1}; and the minimal $(k+1)$-letter word is
$B_k=100\ldots 0$ ($k$ zeros). We have $n(B_k)-n(M_k)=1$ which is equivalent
to (\ref{eq0}).

\head{Proof of (2)} We shall show by induction that the sets $\mathcal{S}_k$
of words of length $k$ in $\mathcal{S}$ are determined uniquely; this is true
for $k=1$ ($\mathcal{S}_1=\{1\}$). Given $\mathcal{S}_k$, the set
$\mathcal{S}_{k+1}$ consists of all words $w0$ for $w\in\mathcal{S}_k$ and,
possibly, some of the words $w1$. Let $w'$ be the immediate successor of $w$
in $\mathcal{S}_k$, or $w'=B_k$ when $w$ is maximal in $\mathcal{S}_k$. We
see that the word $w1$ is in $\mathcal{S}_{k+1}$ if and only if
$n(w'0)-n(w0)=2$, and thus the set $S_{k+1}$ is uniquely determined. Note
that if $n(w'0)-n(w0)$ is less than $1$ or greater than $2$ then the
BTNS does not exist.

\head{Proof of (3)} Let $\mathcal{S}$ be the system defined in the theorem.
We shall need three properties of $\mathcal{S}$ which easily follow
from the definition.

(a) If $w1\in\mathcal{S}$ for some non-empty word $w$
then $w0\in\mathcal{S}$, and in the decomposition
$w0=v_1'\ldots v_s'v$ the word $v$ is empty.

(b) For each $k\geq 1$, the word $M_k$ is the maximal word of length $k$
in $\mathcal{S}$.

(c) For $w\in\mathcal{S}$, the decomposition $w=v_1'\ldots v_s'v$ can be found
in one pass from left to right, without backtracking and/or looking
ahead.

Now, take an arbitrary word $x\in\mathcal{S}$ of length $\geq 2$. We shall find
the word $y\in\mathcal{S}$ immediately preceding $x$ in the lexicographic
order, and check that $n(x)-n(y)=1$ --- this will suffice to prove the claim.

If $x=x'1$ then by (a) we have $x'0\in\mathcal{S}$, therefore $y=x'0$ and
$n(x)-n(y)=1$, as required.

Let $x=x'10\ldots 0$, with $k\geq 1$ zeros at the end.
If $x'$ is empty then (b) implies that $y=M_k$,
and $n(x)-n(y)=1$ from the recurrence (\ref{eq0}).
Otherwise we can apply the property (a) to the word $x'1\in\mathcal{S}$
to obtain that the word $x'0$ is in $\mathcal{S}$, and that in its
decomposition $x'0=v_1'\ldots v_s'v$ the tail $v$ is empty. Therefore,
appending to $x'0$ any word from $\mathcal{S}$ results again in a word from
$\mathcal{S}$, and by (c) all words from $\mathcal{S}$ beginning with
$x'0$ can be obtained in this way.
So, again by (b), we have $y=x'0M_k$. Again the recurrence (\ref{eq0})
implies that $n(x)-n(y)=1$. So, the claim, and the theorem, are proved.

\medbreak

Our three examples have two features in common: first, that the sequence
$M$ is periodic; second, that the numbers $(b_i)$ satisfy some linear
recurrence. The following theorem shows that these two properties are
equivalent.

\begin{theorem}\label{t2}
Let $\mathcal{S}$ be a BTNS. Then its base sequence $B(\mathcal{S})$ satisfies
some linear recurrence if and only if its maximal sequence $M(\mathcal{S})$ is 
periodic. If $M(\mathcal{S})$ has a period of length $p$ after an initial
segment of length $l$ then there exists a linear recurrence for
$B(\mathcal{S})$ of degree at most $p+l$ with integer coefficients.
\end{theorem}

\head{Proof} Let $B(\mathcal{S})=(b_0,b_1,\ldots)$,
$M(\mathcal{S})=m_1m_2\ldots$.
Take any real numbers $a_1,\ldots,a_k$. For $i=1,\ldots,k$ define
\begin{equation}\label{eq1}
x_i=b_{k-i}-\sum_{j=1}^{k-i}a_jb_{k-i-j}.
\end{equation}
In particular, we have $x_k=b_0=1$. Let also
\begin{equation}\label{eq2}
d=a_1+\ldots+a_k-1.
\end{equation}
Equation (\ref{eq0}) implies that, for any $n\geq 0$,
\begin{eqnarray*}
b_{n+k}&-&a_1b_{n+k-1}-a_2b_{n+k-2}-\ldots-a_kb_n=\\
&=&(1+\sum_{i=1}^{n+k}m_ib_{n+k-i})-\\
&&-a_1(1+\sum_{i=1}^{n+k-1}m_ib_{n+k-1-i})-\ldots
-a_k(1+\sum_{i=1}^{n}m_ib_{n-i})\\
&=&\sum_{i=1}^{n}m_i(b_{n+k-i}-a_1b_{n+k-i-1}-\ldots-a_kb_{n-i})+\\
&&+m_{n+1}x_1+\ldots+m_{n+k}x_k-d.
\end{eqnarray*}

It follows that $B(\mathcal{S})$ satisfies the recurrence
$b_{n+k}=a_1b_{n+k-1}+\ldots+a_kb_n$ if and only if for every subword
$(m_{n+1}\ldots m_{n+k})$ of $M(\mathcal{S})$ we have
\begin{equation}\label{eq3}
m_{n+1}x_1+\ldots+m_{n+k}x_k=d.
\end{equation}

If (\ref{eq3}) holds then $m_{n+k}$ is uniquely determined by
$m_{n+1},\ldots,m_{n+k-1}$. As there are only finitely many binary words
of length $k-1$, the sequence $M(\mathcal{S})$ is periodic.

Conversely, let $M(\mathcal{S})$ be periodic with period of length $p$
after an initial sequence of length $l$. Take $k=l+p$. We can easily
satisfy the equations (\ref{eq3}) by taking $x_1=\ldots=x_l=0$,
$x_{l+1}=\ldots=x_{l+p}=1$, and $d$ equal to the number of ones
in the period. Then from the equations (\ref{eq1}) we can successively
determine the numbers $a_1,\ldots,a_{k-1}$:
$$a_{k-i}=b_{k-i}-x_i-\sum_{j=1}^{k-i-1}a_jb_{k-i-j}.$$
Finally, from the equation (\ref{eq2}) we can find $a_k$.
The theorem is proved.

When $M(\mathcal{S})$ is purely periodic with period of length $p$, the
recurrence for $B(\mathcal{S})$ is especially simple: we can take
$$a_1=m_1,\;a_2=m_2,\ldots,a_{p-1}=m_{p-1},\;a_p=m_p+1.$$

$l+p$ is not necessarily the lowest possible degree of the recurrence;
an interesting question is: how long can the period of $M(\mathcal{S})$ be
if $B(\mathcal{S})$ satisfies a recurrence of degree $k$?

\section{Complete tables}

The most interesting property of the Fibonacci table is its 
{\em completeness}: it contains all sequences satisfying Fibonacci
recurrence (Theorem \ref{th3_1}). In this section we shall give another,
purely combinatorial proof of this fact, and at the same time we shall
find infinitely many other recurrences having the same nice property.

\begin{definition}
Let $\mathcal{S}$ be a BTNS whose base sequence $B=(b_0,b_1,b_2,\ldots)$
satisfies the recurrence
\begin{equation}\label{eqA}
b_{n+k}=a_1b_{n+k-1}+\ldots+a_kb_n.
\end{equation}
The system $\mathcal{S}$ is complete if every sequence of positive integers
satisfying this recurrence occurs, from some point on, in the successor 
table of $\mathcal{S}$.
\end{definition}

One way to demonstrate that a certain $\mathcal{S}$ is complete is the 
following: to prove that every sequence of natural numbers satisfying
(\ref{eqA}) is a linear combination with positive integer coefficients
of some successor sequences (for instance, of some shifts of the base 
sequence); and then to prove that every such linear combination is
a successor sequence from some point on. Note that the 
base sequence always coincides with the first row of the successor 
table. 

Let us do this for the BTNS with the maximal sequence 
$M(\mathcal{S})=(11\ldots 10)^*$ ($k-1$ ones). The system $\mathcal{S}$
consists of all binary words not containing $k$ consecutive ones;
its base sequence satisfies the recurrence 
\begin{equation}\label{eqB}
b_{n+k}=b_{n+k-1}+\ldots+b_n
\end{equation}
with the initial values $b_i=2^i$ for $i=0,1,\ldots,k-1$. 
Going backwards, we find:
$$b_{-1}=1,\;b_{-2}=\ldots=b_{-k}=0,\;b_{-(k+1)}=1.$$
Thus, the sequences satisfying (\ref{eqB}) and beginning with $k$-tuples
\begin{eqnarray*}
b^{(0)}&=&(1,0,0,\ldots,0,0)\\
b^{(1)}&=&(0,1,1,2,\ldots,2^{k-3})\\
&&\ldots\\
b^{(k-3)}&=&(0,\ldots,0,1,1,2)\\
b^{(k-2)}&=&(0,0,\ldots,0,1,1)\\
b^{(k-1)}&=&(0,0,\ldots,0,0,1)
\end{eqnarray*}
are shifts of the base sequence.

The vectors $b^{(0)},b^{(1)},\ldots,b^{(k-1)}$ form a basis of the 
$k$-dimensional row space. One easily finds that
\begin{eqnarray*}
(a_0,\ldots,a_{k-1})&=&a_0b^{(0)}+a_1b^{(1)}+(a_2-a_1)b^{(2)}+
(a_3-a_2-a_1)b^{(3)}\\&&+\ldots+(a_{k-1}-a_{k-2}-\ldots-a_1)b^{(k-1)}.
\end{eqnarray*}
Thus, if $(x_n)_{n\geq 0}$ is any sequence of natural numbers satisfying
(\ref{eqB}) then the $k$-tuple $(x_k,x_{k+1},\ldots,x_{2k-1})$ is a linear
combination of $b^{(0)},\ldots,b^{(k-1)}$ with integer non-negative
coefficients; and we have fulfilled the first part of our plan.

\medskip
For the second part, we introduce the following game. Fix a natural number 
$k\geq 2$.
On the doubly infinite strip of squares indexed by integers 
are placed finitely many pebbles (possibly, more than one pebble in a square).
One is allowed to make moves of two kinds.

1. If there are $k$ consecutive non-empty squares, say $n,n+1,\ldots,n+k-1$
then one can remove one pebble from each of them, and add one pebble to the 
square $n+k$.

2. If a square $n$ contains two or more  pebbles then one can remove from 
it two pebbles, and add to the squares $n-k$ and $n+1$ one pebble each. 

\begin{lemma}\label{pebblem}
The game described above always terminates; and the final position 
depends only on the initial position, and not on the sequence of moves.
\end{lemma}

\head{Proof}
Let $(w_i)_{i\in\Z}$ be any sequence of real numbers satisfying the recurrence
(\ref{eqB}). Every position in the game is determined by the sequence
$(a_i)$ of non-negative integers, all but finitely many of which are 
equal to 0. The weight of the position $A=(a_i)$ is defined as 
$$w(A)=\sum w_ia_i.$$
The rules of the game are chosen so that legal moves don't change the weight 
of the position. Also, they don't increase the number of pebbles.

Now, let $\alpha$ be the positive root of the equation
$$x^k=x^{k-1}+x^{k-2}+\ldots+x+1.$$
We have $1<\alpha<2$. Let $w_i=\alpha^i$; obviously, this sequence 
satisfies (\ref{eqB}). 

First we shall prove by induction on the number of pebbles that the game 
eventually stops. Consider the maximal index of a pebble in the position.
It does not decrease, but it cannot increase indefinitely (the weight 
of the position is bounded) --- therefore, from some moment on, the pebble
with the maximal index is left untouched, and we can apply induction to the 
remaining pebbles.

In any final position, there is at most one pebble in each square, and 
there are no $k$ consecutive non-empty squares. To finish the proof 
it suffices to show that any two different positions with these 
properties have different weights. Let $X=(x_i)$ and $Y=(y_i)$ be two 
such positions; $n$ --- the maximal index for which $x_n\neq y_n$; say,
$x_n=1$, $y_n=0$. Let $s=\sum_{i>n}w_ix_i=\sum_{i>n}w_iy_i$. Then 
\begin{eqnarray*}
w(X)&\geq&s+\alpha^n;\\
w(Y)&  < &s+\sum_{l\geq 0}\sum_{i=1}^{k-1}\alpha^{n-kl-i}=s+\alpha^n;
\end{eqnarray*}
and $w(X)>w(Y)$. The lemma is proved. 

\medskip
Now, let $a_n=\sum x_ib_{n+i}$ be any linear combination of shifts 
of the base sequence with non-negative integer coefficients.
Take $(x_i)$ as the initial position of the above game; let $(y_i)$
be the corresponding unique final position. As in the proof of the lemma,
we have $\sum y_ib_{n+i}=\sum x_ib_{n+i}=a_n$. 
Let $m$ be the minimal index for which $y_m\neq 0$. We see that, starting from
$n=-m$, the sequence $(a_i)$ is a successor sequence.

\medskip
By a similar but more involved argument one can prove that the numeration
system with the maximal sequence $100100100\ldots$ (and with the recurrence
$b_{n+1}=b_n+b_{n-2}$) is also complete. On the other hand, this is not so
for the recurrence $b_{n+1}=b_n+b_{n-3}$. 

\medskip
{\bf Problem.}\quad Classify all based tree-like numeration systems
with recurrent base sequences which are complete.
 
\medskip
{\bf Remark.} Lemma \ref{pebblem} was given as a problem at 1997 Russian
Mathematical Olympiad, and was voted by the participants 
``the best problem of the year''.

\section{Exercise}

Given $n$, form all possible sequences of
positive integers with sum $n$. For each such sequence, multiply
the terms together; then take the sum of all these products.
What is the result?

\begin{thebibliography}{9}

\bibitem{coxeter}
H. S. M. Coxeter, \textit{Introduction to Geometry}, Wiley, New York, 1961.

\bibitem{kimberling}
C. Kimberling,
Numeration systems and fractal sequences,
\textit{Acta Arith.} \textbf{73} (1995), 103--117.

\end{thebibliography}

\end{document}




